\chapter{Introduction}
\section{Background and Context}
The use of embedded systems is in current times more and more ubiquitous and is used in a wide range of applications. 
Mostly, but not solely, known for their use in the automotive industry, embedded systems are used on a daily by many people.
Since the uses are often in situations in which the system has to react in real-time, the systems have to be designed with a high level of reliability and efficiency.
This is especially true for systems that are used in the automotive industry, where the system has to react in real-time to the environment and has to be able to make decisions in a fraction of a second.
The use of embedded systems in the automotive industry is not only limited to the control of the engine or the transmission, but also includes the control of the infotainment system, the climate control, and the driver assistance systems.
To ensure that the system is able to react in real-time, the system has to be designed to be able to handle the required tasks in a timely manner.
The scheduling of the tasks need to consider the longest run times of given tasks, also known as the \ac{WCET}.
The \ac{WCET} is the maximum time that a task needs to execute on the system.
Determining the \ac{WCET} is a complex task to be done in different ways.
One way is the use of static analysis, which is done by analyzing the source code of the system.
Another way is the use of dynamic analysis, which is done by executing the system and measuring the time that the system needs to execute the task.
Either way the analysis has to be checked for its correctness and the results have to be validated.
To be able to do this, the analysis methods are tested against a set of benchmarks, which are designed in a way that the \ac{WCET} is known.
Each benchmark has it's own characteristics and is used to test the analysis methods for dealing with these characteristics and patterns.
For example do many benchmarks have their own limitation in the aspects of operating systems, compilers and system configurations. \cite{falk_taclebench_2016}

\subsection{RTOS}


\subsection{\ac{WCET} and \ac{WCRT}}


\subsection{benchmarks}


\section{Problem Statement}
Existing tools often do lack some parts of occurring problems in the real life.
For the most parts of analysis the  
TASKers \cite{eichler_taskers_2018} for example is built in a way to only construct a single chain of tasks in the resulting task set.
The resulting task set is the sum of all \ac{WCET} of each task in the chain.
Knowing all system specifications by analyzing code is not possible	\cite{}


\section{Research Objectives}
This work targets this weaknesses with the goal to add a new approach to the existing concepts to tackle the problem of generating systems with known response times.
In order to overcome known problems the plan is to create the resulting system configuration from the other end

\section{Research Questions}
This thesis plans to tackle the generation of systems with known task sets and thus known \ac{WCET} and \ac{WCRT} for each task.
Besides that the generator shall create an opportunity for flexible variations in generating different systems.
To create diversity the generation 

\section{Structure of the Thesis}

\todo{insight in the following chapters}
