\chapter{Approach}
\section{System-model}
\label{sec:model}
In this section, we define the system model that serves as the foundation for the concept of the generated task-sets.

\subsection{Task Model}
\label{sec:task_model}
Part of the system model is a harmonic set of periodic tasks $\Tau=\{\Tau_i: i = 1, 2, \ldots, n\}$.
Due to being a harmonic task-set and its properties (see \cref{sec:scheduling}) the \textit{planning cycle}\cite{dar-tzenpengAssignmentSchedulingCommunicating1997} of $\Tau$ is defined by the \textit{\ac{LCM}} of all tasks' periods $\{P_1, \ldots, P_n\}$, resulting in the period $P_{max}$ of the task with the highest period.
Each task $\Tau_i$ is characterized by its period $P_i$, its list of possible execution times $\{CP_{i_1}, CP_{i_2}, \ldots\}$, and the set of related jobs $\tau_i=\{\tau_{i_0}, \tau_{i_1}, \dots, \tau_{i_m}; m = \frac{P_{max}}{P_i} - 1\}$.

Each job $\tau_{i_j}$ is described by the release time $r_{i_m}$, the implicit deadline of $d_{i_m} = r_{i_m} + P_i$ as the next period $P_i+1$ begins, the jobs computational time $C_{i_m}$

\subsection{Hardware Model}
\label{sec:hardware_model}
The generated model is designed to be a meta model with no constraints on hardware specifications, making it hardware independent.
No further details about the hardware are provided.

\subsection{Execution Model}
\label{sec:execution_model}
In the execution model, a preemptive scheduler using \ac{RMS} (see \cref{sec:scheduling}) is employed to manage the periodic tasks. 
The scheduler operates by assigning each task $\Tau_i$ a priority inversely proportional to its period $P_i$ \ref{liuSchedulingAlgorithmsMultiprogramming1973}.
Tasks with shorter periods receive higher priorities and are scheduled more frequently.
Higher-priority tasks can preempt lower-priority tasks.
If a higher-priority task becomes ready to execute while a lower-priority task is running, the scheduler will interrupt the lower-priority task and allocate the processor to the higher-priority task.
The generated schedule is represented by a list of time frames associated to a job being run in that time frame $sched = {begin, end, \tau_{i_j}}$.

Making use of the \ac{RMS} in combination with the in \cref{sec:task_model} mentioned harmonic task-set it is possible to reach utilizations up to $1$ \cite{liuSchedulingAlgorithmsMultiprogramming1973}.


\cite{dar-tzenpengAssignmentSchedulingCommunicating1997}


\section{Concept}
\label{sec:concept}
The general workflow involves several key steps. 
\begin{enumerate}
	\item Generate period times based on the server model.
	\item Create task chains and identify any missing links within these chains.
	\item Distribute the load across each period and generate tasks with lower execution times to balance the workload.
	\item Connect tasks and create conditions, considering the idea of workloads with resources attached, such as inter-task communication.
	\item Spread the server model into multiple server instances.
	\item Build a scheduler and perform a schedulability test using the tree structure of the previously built server instances.
	\item Export the generated task-set for further use.
\end{enumerate}
% how to generate period times (serverModel)

% generate chains
% missing links in chain

% spread load per period

% generate load with lower execution times

% connecting tasks, creating conditions
% idea of workloads with resources attached (inter-task communication)

% spread serverModel into serverInstances

% building a scheduler (schedulability test)
% using tree structure of previous build server instances

% exporting task-set 

\section{Implementation}
\label{sec:implementation}


\section{Tasks with Release Times and Execution Times}
\label{sec:tasks_release_execution}

In real-time systems, tasks are characterized by their release times and execution times.
The release time of a task is the time at which the task becomes ready for execution.
Execution times can vary depending on the complexity and requirements of the task.

Consider a set of tasks \( \tau = \{\tau_1, \tau_2, \ldots, \tau_n\} \).
Each task \( \tau_i \) has a release time \( r_i \) and a list of possible execution times \( \{e_{i1}, e_{i2}, \ldots, e_{im}\} \).
The release time \( r_i \) indicates when the task \( \tau_i \) is ready to be executed, and the list of execution times represents the different durations the task might take to complete under various conditions.
The release times will be constrained further in the section \cref{sec:harmonic_task_set}.

For example, let us define three tasks with their respective release times and execution times:

\begin{itemize}
	\item Task \( \tau_1 \): Release time \( r_1 = 0 \), Execution times \( \{3, 4, 5\} \)
	\item Task \( \tau_2 \): Release time \( r_2 = 2 \), Execution times \( \{2, 3\} \)
	\item Task \( \tau_3 \): Release time \( r_3 = 4 \), Execution times \( \{1, 2, 3\} \)
\end{itemize}

The scheduler must consider both the release times and the possible execution times to ensure that all tasks meet their deadlines.
Each task will be assigned a hard deadline which strictly encases the time frame a task may be scheduled in. 
This involves determining the optimal order of execution and allocating sufficient resources to handle the variability in execution times.

By analyzing the release times and execution times, we can develop strategies to improve the efficiency and reliability of real-time systems, ensuring that tasks are completed within their required time frames.

\section{Flow of Data}
\label{sec:flow_of_data}

\section{Creating Conditions}
\label{sec:creating_conditions}

\section{Representation}
\label{sec:representation}