\chapter{Approach}
\section{Model}
\label{sec:model}
periodic tasks
hard deadlines
preemption
inter-task communication


\section{Tasks with Release Times and Execution Times}
\label{sec:tasks_release_execution}

In real-time systems, tasks are characterized by their release times and execution times.
The release time of a task is the time at which the task becomes ready for execution.
Execution times can vary depending on the complexity and requirements of the task.

Consider a set of tasks \( \tau = \{\tau_1, \tau_2, \ldots, \tau_n\} \).
Each task \( \tau_i \) has a release time \( r_i \) and a list of possible execution times \( \{e_{i1}, e_{i2}, \ldots, e_{im}\} \).
The release time \( r_i \) indicates when the task \( \tau_i \) is ready to be executed, and the list of execution times represents the different durations the task might take to complete under various conditions.
The release times will be constrained further in the section \cref{sec:harmonic_task_set}.

For example, let us define three tasks with their respective release times and execution times:

\begin{itemize}
	\item Task \( \tau_1 \): Release time \( r_1 = 0 \), Execution times \( \{3, 4, 5\} \)
	\item Task \( \tau_2 \): Release time \( r_2 = 2 \), Execution times \( \{2, 3\} \)
	\item Task \( \tau_3 \): Release time \( r_3 = 4 \), Execution times \( \{1, 2, 3\} \)
\end{itemize}

The scheduler must consider both the release times and the possible execution times to ensure that all tasks meet their deadlines.
Each task will be assigned a hard deadline which strictly encases the time frame a task may be scheduled in. 
This involves determining the optimal order of execution and allocating sufficient resources to handle the variability in execution times.

By analyzing the release times and execution times, we can develop strategies to improve the efficiency and reliability of real-time systems, ensuring that tasks are completed within their required time frames.

\section{Flow of Data}
\label{sec:flow_of_data}

\section{Creating Conditions}
\label{sec:creating_conditions}

\section{Representation}
\label{sec:representation}